\documentclass[10pt]{standalone}
% load the "Hochschule Bonn-Rhein-Sieg" beamer theme
%\usetheme{HBRS}

%%%%%%%%%%%%%%%%%%%%%%%%%%%%%%%%%%%%%%%%%%%%%%%%%%%%%%%%%%%%%%%%%%%%%%
% packages (some, like tikz, are already loaded with the HBRS theme)

%Fonts and encoding:
\usepackage{lmodern}
\usepackage[utf8]{inputenc}
\usepackage[ngerman]{babel}

%used for the newdateformat command
\usepackage{datetime}

%multimedia support (!! depends on pdf viewer !!)
\usepackage{media9}

%floating point operations
\usepackage{fp}

%"TikZ ist kein Zeichenprogramm"
\usepackage{tikz}
\usetikzlibrary{arrows,
	calc,
	patterns,
	fit,
	external}
\usepackage{pgfplots}
\pgfplotsset{compat=newest}
\usepgfplotslibrary{fillbetween}

%electric circuits using tikz
\usepackage[europeanvoltages]{circuitikz}


\usepackage{color,colortbl}
\definecolor{darkgreen}{rgb}{.0,.5,.0} 
\definecolor{lightgray}{rgb}{.8,.8,.8} 
\definecolor{lightgreen}{rgb}{.514,.804,.737}

\usepackage{tabularx}
\newcolumntype{C}[1]{>{\centering\let\newline\\\arraybackslash\hspace{0pt}}m{#1}}
\newcolumntype{L}[1]{>{\let\newline\\\arraybackslash\hspace{0pt}}m{#1}}


\definecolor{darkgreen}{rgb}{0,0.67,0}
\definecolor{DG}{rgb}{0.3,0.3,0.3}
\definecolor{MG}{rgb}{0.6,0.6,0.6}
\definecolor{LG}{rgb}{0.9,0.9,0.9}

\usepackage[absolute]{textpos}


\usepackage{listings}
\definecolor{matlabgreen}{RGB}{28,172,0} % color values Red, Green, Blue
\definecolor{matlablilas}{RGB}{170,55,241}
\lstset{
	language=Matlab,
	emph=[1]{for,while,end,break,function},emphstyle=[1]\color{blue}, %some words to emphasise
	commentstyle=\color{matlabgreen},
	showstringspaces=false
}


\usepackage{units} % for nicefrac

%%%%%%%%%%%%%%%%%%%%%%%%%%%%%%%%%%%%%%%%%%%%%%%%%%%%%%%%%%%%%%%%%%%%%%
% title, subtitle,author,date and optional footlinetext

\title{Modellbildung und Simulation 1}
%\subtitle{Sommersemester 2019}
\author{Jan Kleinert}
\date{FB 03 -- EMT}

%\footlinetext{%
	%\insertframenumber/\inserttotalframenumber%
	%\newline%
	%ModSim1, Jan Kleinert%
	%\newline%
	%Sommersemester 2019%
	%}



%%%%%%%%%%%%%%%%%%%%%%%%%%%%%%%%%%%%%%%%%%%%%%%%%%%%%%%%%%%%%%%%%%%%%%

% qr code (default units for width/height are inches) 
%\usepackage{qrcode}

\makeatletter 
\newcommand{\qr}[2]{% 
	\href{#2}{% 
		\begin{tikzpicture} 
			\node [rectangle,draw=HBRSblue,rounded corners=5pt, line width=2pt, fill=HBRSlightgray]{%
				\qrcode{#2}% 
				\begin{tabular}{l} \textcolor{HBRSblue}{#1}\\\\
					\footnotesize{\texttt{\textcolor{HBRSblue}{#2}}} 
				\end{tabular} 
			}; 
		\end{tikzpicture} 
	} 
} 
\makeatother

\begin{document}
	
\begin{tikzpicture}
	\begin{axis}[ 
		height=4cm,
		width=5cm, 
		axis lines=middle,
		clip=false,
		axis line style={->},
		% xlabel={$x$},
		% ylabel={$y$},
		ticks=none,
		xmin=-0.25,
		xmax=1.25,
		ymin=-0.5,
		ymax=3
		]
		
		
		\addplot[red,samples=500,domain=-0.25:1.25,line width = 1pt] {exp(x)};
		
		
%		\addplot[name path=p,blue,samples=500,domain=0.4:0.8, line width = 1 pt] { 1.0506 + 0.7374*x + 0.9141*x^2};
%		\path[name path=axis] (axis cs:0.4,0) -- (axis cs:0.8,0);
%		\addplot [
%		thick,
%		color=blue,
%		fill=blue, 
%		fill opacity=0.15
%		]
%		fill between[
%		of=p and axis,
%		%         soft clip={domain=0.399:0.8},
%		];
		
		%\draw[blue, line width=1pt] (0.0,1) -- (0.4,1.4918)% -- (0.4,0) -- (0.4,1.4918);
		
		%\node at (0.0,1) [draw,circle,scale=0.25,fill=black] {};
		\node at (0.4,1.4918) [draw,circle,scale=0.25,fill=black] {};
		%\node at (0.6,1.8221) [draw,circle,scale=0.25,fill=black] {};
		\node at (0.8,2.2255) [draw,circle,scale=0.25,fill=black] {};
		
		\draw[dashed] (0.0,0) node[below] {$x_{i-1}$} -- (0.0,3);
		\draw[dashed] (0.4,0) node[below] {$x_{i}$} -- (0.4,3);
		\draw[dashed] (0.8,0) node[below] {$x_{i+1}$} -- (0.8,3);
		
		\draw[blue, line width=1pt,dashed] (0.4,1.4918) -- (0.8,1.4918);
		\draw[blue, line width=1pt,dashed] (0.8,1.4918) node[right] {$\Delta y$} -- (0.8,2.2255);
		\node at (0.6,1.4918) [blue,below] {$\Delta x$};
		
		\addplot[blue,samples=500,domain=-0.25:1.25,line width = 1pt] {(2.2255-1.4918)/0.4*x+(1.4918-(2.2255-1.4918))};
		
		\node at (0.4,3) [blue,above] {$y'(x_i)\approx \frac{\Delta y}{\Delta x} = \frac{y(x_{i+1})-y(x_{i})}{x_{i+1} - x_{i}}$};
		
	\end{axis}
	
	
\end{tikzpicture}

\end{document}
